% ═══════════════════════════════════════════════════════════════════════
% Algebraic Rigidity and Topological Spin Asymmetry
% in Legendre Intervals
% Titan Project — Paper IX — February 2026
% ═══════════════════════════════════════════════════════════════════════

\documentclass[11pt, a4paper]{article}

\usepackage[top=28mm, bottom=28mm, left=25mm, right=25mm]{geometry}
\usepackage[T1]{fontenc}
\usepackage{amsmath, amssymb, amsthm, mathtools}
\usepackage{mathrsfs}
\usepackage{graphicx}
\usepackage[dvipsnames]{xcolor}
\usepackage{enumitem}
\usepackage{booktabs}
\usepackage{hyperref}
\usepackage{float}

\newtheorem{theorem}{Theorem}[section]
\newtheorem{lemma}[theorem]{Lemma}
\newtheorem{proposition}[theorem]{Proposition}
\newtheorem{corollary}[theorem]{Corollary}
\newtheorem{conjecture}[theorem]{Conjecture}
\theoremstyle{definition}
\newtheorem{definition}[theorem]{Definition}
\newtheorem{example}[theorem]{Example}
\theoremstyle{remark}
\newtheorem{remark}[theorem]{Remark}

\newcommand{\Leg}[2]{\left(\frac{#1}{#2}\right)}
\newcommand{\rad}{\mathrm{rad}}
\newcommand{\QR}{\mathrm{QR}}
\newcommand{\NR}{\mathrm{NR}}

\title{\textbf{Algebraic Rigidity and Quadratic Residue \\
Asymmetry in Legendre Intervals}}
\author{Ruqing Chen\\[4pt]
\small GUT Geoservice Inc., Montr\'eal, Canada\\[2pt]
\small \texttt{ruqing@hotmail.com}\\[2pt]
\small Repository: \url{https://github.com/Ruqing1963/legendre-spin-asymmetry}}
\date{February 2026}

\begin{document}

\maketitle

\begin{abstract}
We prove that Legendre intervals $\mathcal{I}_n = [(n-1)^2, n^2]$
whose length $p = 2n - 1$ is prime possess a rigid algebraic
structure: the interior integers
$\mathcal{I}_n^\circ = \{(n-1)^2 + 1, \ldots, n^2 - 1\}$
form a \emph{punctured complete residue system} modulo $p$,
missing exactly the quadratic residue class $r \equiv n^2 \pmod{p}$.
As a consequence, the Legendre symbol distribution over
$\mathcal{I}_n^\circ$ exhibits a universal \emph{spin asymmetry}:
the number of quadratic non-residues exceeds the number
of quadratic residues by exactly $1$.
This theorem is verified computationally for all qualifying
$n \le 2000$ (549 instances, 100\% confirmation).
We complement this algebraic result with empirical measurements
of the \emph{discrete Wronskian volume}---the radical compression
ratio of maximal prime-free subsequences near $n^2$---and observe
that this ratio stabilizes near $1.20$ as $n \to \infty$,
well below the Szpiro threshold.
Together, these findings reveal previously unobserved algebraic
and arithmetic rigidity constraints on the internal structure
of Legendre intervals.
\end{abstract}

\bigskip

% ═══════════════════════════════════════════════════════════════════════
\section{Introduction}\label{sec:intro}
% ═══════════════════════════════════════════════════════════════════════

\subsection{Legendre's Conjecture and interval structure}

Legendre's Conjecture asserts that for every $n \ge 1$,
the interval $[n^2, (n+1)^2]$ contains at least one prime.
Equivalently (by a shift of index), for every $n \ge 2$,
the interval $\mathcal{I}_n = [(n-1)^2, n^2]$ contains a prime.
While the Prime Number Theorem predicts approximately
$n / \log n$ primes in $\mathcal{I}_n$, no unconditional
proof of even a single prime has been achieved.

Most approaches to Legendre's Conjecture focus on
\emph{counting} primes via sieve methods or zero-free regions
of the Riemann zeta function.
In this paper, we take a fundamentally different approach:
we study the \emph{algebraic structure} of the interval itself,
specifically its behavior as a residue system modulo its own
length $p = 2n - 1$.

\subsection{Main result: the Spin Asymmetry Theorem}

When $p = 2n - 1$ is prime, the $p - 1$ integers in the
interior of $\mathcal{I}_n$ form a remarkable algebraic object:
a complete residue system modulo $p$ with exactly one class
removed.
We prove (Theorem~\ref{thm:main}) that the removed class
is \emph{always} a quadratic residue, creating a universal
asymmetry in the Legendre symbol distribution.

This asymmetry is not a statistical phenomenon---it is an
\emph{exact algebraic identity} that holds for every $n$
with $2n - 1$ prime.

\subsection{Complementary evidence: Wronskian compression}

We also study the multiplicative structure of maximal
prime-free subsequences (``prime vacuums'') within
$\mathcal{I}_n$ by computing the ratio of the logarithmic
volume $\log \prod C_i$ to the logarithmic radical
$\log \rad(\prod C_i)$ for sequences of consecutive composites.
This ratio, which we call the \emph{discrete Wronskian
compression}, measures the degree to which prime factors
are shared among consecutive composites.
We observe empirically that this ratio stabilizes
near $1.20$ across six orders of magnitude in $n$.

\subsection{Context within the Titan Project}

In a companion paper \cite{ChenFF}, we proved the function
field analogue of Legendre's Conjecture via monodromy and
the Chebotarev density theorem.
The algebraic structure uncovered here---quadratic residue
asymmetry governed by the interval length---has no direct
analogue in the function field setting, where the parameter
space is an affine variety rather than a residue system.
The present paper thus provides a complementary,
specifically integer-theoretic perspective on the problem.
The conductor rigidity phenomena studied in
\cite{ChenConductor, ChenLandau, ChenAP}
established that classical $L$-function methods encounter
structural barriers for additive prime number problems.
The quadratic residue framework developed here may offer
a route that circumvents these barriers.

% ═══════════════════════════════════════════════════════════════════════
\section{The Spin Asymmetry Theorem}\label{sec:theorem}
% ═══════════════════════════════════════════════════════════════════════

\subsection{Setup and notation}

\begin{definition}[Legendre interval]\label{def:interval}
For $n \ge 2$, the \emph{Legendre interval} is
$\mathcal{I}_n = [(n-1)^2, \, n^2]$.
Its \emph{interior} is
$\mathcal{I}_n^\circ = \{(n-1)^2 + 1, \ldots, n^2 - 1\}$.
The interval has total length $|\mathcal{I}_n| = 2n - 1$
(inclusive of endpoints) and interior cardinality
$|\mathcal{I}_n^\circ| = 2n - 3$.
\end{definition}

\begin{remark}\label{rem:length}
The interval $[(n-1)^2, n^2]$ contains $n^2 - (n-1)^2 + 1 = 2n$
integers (inclusive of both endpoints).
If $p := 2n - 1$ is prime, then $|\mathcal{I}_n| = p + 1 = 2n$
and $|\mathcal{I}_n^\circ| = p - 1 = 2n - 2$.
\end{remark}

\begin{definition}[Spin]\label{def:spin}
Let $p$ be an odd prime. For an integer $a$, the \emph{spin}
of $a$ modulo $p$ is the Legendre symbol
$\Leg{a}{p} \in \{-1, 0, +1\}$.
We say $a$ has \emph{positive spin} if $\Leg{a}{p} = +1$
(quadratic residue), \emph{negative spin} if
$\Leg{a}{p} = -1$ (quadratic non-residue), and
\emph{zero spin} if $p \mid a$.
\end{definition}

\subsection{The punctured residue system}

\begin{lemma}\label{lem:residues}
Let $n \ge 2$ and $p = 2n - 1$.
Set $r \equiv n^2 \pmod{p}$.
Then:
\begin{enumerate}[nosep, label=(\roman*)]
\item $(n-1)^2 \equiv n^2 \equiv r \pmod{p}$.
\item The interior $\mathcal{I}_n^\circ$ consists of
$p - 1$ consecutive integers, and its elements represent
every residue class modulo $p$ \emph{except} $r$.
In particular, exactly one element of $\mathcal{I}_n^\circ$
is divisible by $p$.
\item $\gcd(n, p) = 1$, so
$r \equiv n^2 \pmod{p}$ is a nonzero quadratic residue:
$\Leg{r}{p} = +1$.
\end{enumerate}
\end{lemma}

\begin{proof}
\textbf{(i)} $(n-1)^2 = n^2 - 2n + 1 = n^2 - (2n-1) = n^2 - p$,
so $(n-1)^2 \equiv n^2 \pmod{p}$.

\textbf{(ii)} The set
$\{(n-1)^2, (n-1)^2 + 1, \ldots, (n-1)^2 + (p-1)\}$
$= \{(n-1)^2, \ldots, n^2 - 1\}$
has exactly $p$ elements (since $(n-1)^2 + (p-1) = n^2 - 1$),
hence covers every residue class modulo $p$ exactly once.
Removing the first element $(n-1)^2$ (which has residue $r$)
yields $\mathcal{I}_n^\circ = \{(n-1)^2 + 1, \ldots, n^2 - 1\}$,
which covers all residues \emph{except} $r$.
Since $r \ne 0$ (see (iii)), the class $0$ is represented,
so exactly one element of $\mathcal{I}_n^\circ$ is divisible by $p$.

\textbf{(iii)} $\gcd(n, 2n-1) = \gcd(n, -1) = 1$,
so $p \nmid n$, hence $r = n^2 \bmod p \ne 0$.
Since $r \equiv n^2 \pmod{p}$ is explicitly a perfect square
modulo $p$, we have $\Leg{r}{p} = +1$.
\end{proof}

\subsection{The main theorem}

\begin{theorem}[Spin Asymmetry]\label{thm:main}
Let $n \ge 2$ be an integer such that $p = 2n - 1$ is prime.
Denote by $N^+$, $N^-$, $N^0$ the numbers of elements of
$\mathcal{I}_n^\circ$ with positive, negative, and zero
spin modulo $p$, respectively.
Then:
\begin{equation}\label{eq:asymmetry}
    N^+ = \frac{p-3}{2} = n - 2,
    \qquad
    N^- = \frac{p-1}{2} = n - 1,
    \qquad
    N^0 = 1.
\end{equation}
In particular,
\begin{equation}\label{eq:excess}
    N^- - N^+ = 1
\end{equation}
for every such $n$. Quadratic non-residues always
exceed quadratic residues by exactly one.
\end{theorem}

\begin{proof}
By Lemma~\ref{lem:residues}(ii), the elements of
$\mathcal{I}_n^\circ$ represent all residue classes
modulo $p$ except $r$.
Among the $p$ residue classes modulo $p$:
\begin{itemize}[nosep]
\item Class $0$: spin $0$. Count: $1$.
\item Nonzero QR classes: spin $+1$.
Count: $(p-1)/2$.
\item Nonzero NR classes: spin $-1$.
Count: $(p-1)/2$.
\end{itemize}
Total: $1 + (p-1)/2 + (p-1)/2 = p$.

The interior $\mathcal{I}_n^\circ$ covers all classes
\emph{except} $r$, which is a nonzero QR
(Lemma~\ref{lem:residues}(iii)).
Therefore:
\begin{align*}
    N^0 &= 1, \\
    N^+ &= \frac{p-1}{2} - 1 = \frac{p-3}{2} = n - 2, \\
    N^- &= \frac{p-1}{2} = n - 1.
\end{align*}
Hence $N^- - N^+ = (n-1) - (n-2) = 1$.
\end{proof}

\begin{remark}
The proof is entirely elementary, using only the
structure of quadratic residues modulo a prime and the
coincidence of endpoints modulo $p$.
The key arithmetic input is the identity
$(n-1)^2 \equiv n^2 \pmod{2n-1}$,
which is trivially verified.
\end{remark}

\subsection{Computational verification}

\begin{proposition}\label{prop:verification}
The Spin Asymmetry Theorem has been verified computationally
for all $n \le 2000$ such that $2n - 1$ is prime
(549 instances). In every case,
$N^- - N^+ = 1$ and $N^0 = 1$.
\end{proposition}

\begin{example}\label{ex:small}
For $n = 4$, $p = 7$, $\mathcal{I}_4^\circ = \{10, 11, 12,
13, 14, 15\}$.
The residues modulo $7$ are $\{3, 4, 5, 6, 0, 1\}$,
missing $r = 16 \bmod 7 = 2$.
The QRs modulo $7$ are $\{1, 2, 4\}$; NRs are $\{3, 5, 6\}$.
Among the represented nonzero residues $\{1, 3, 4, 5, 6\}$:
QRs $= \{1, 4\}$ ($N^+ = 2$), NRs $= \{3, 5, 6\}$ ($N^- = 3$),
plus $\{0\}$ ($N^0 = 1$). Indeed $N^- - N^+ = 1$.
\end{example}

\begin{example}\label{ex:large}
For $n = 100$, $p = 199$, $\mathcal{I}_{100}^\circ$
has $198$ elements covering all residues modulo $199$
except $r = 10000 \bmod 199 = 50$.
Since $50$ is a QR modulo $199$ (as $50 \equiv 100^2
\pmod{199}$):
$N^+ = 98$, $N^- = 99$, $N^0 = 1$.
\end{example}

% ═══════════════════════════════════════════════════════════════════════
\section{Structural Implications}\label{sec:implications}
% ═══════════════════════════════════════════════════════════════════════

\subsection{The asymmetry as an algebraic constraint}

The Spin Asymmetry Theorem reveals that when $p = 2n - 1$
is prime, the Legendre interval is not merely a collection
of consecutive integers: it is a structured algebraic object
whose internal residue distribution is \emph{exactly}
determined by $n$.

This structure constrains any prime-free realization of the
interval.
If all elements of $\mathcal{I}_n$ were composite,
these $p - 1$ composites would need to simultaneously satisfy:
\begin{enumerate}[nosep, label=(\alph*)]
\item \emph{Divisibility:} every element has at least two
prime factors (counted with multiplicity).
\item \emph{Residue coverage:} the composites cover all
residue classes modulo $p$ except $r$, including exactly
one multiple of $p$ itself.
\item \emph{Spin asymmetry:} the Legendre symbols of these
composites reproduce the exact $(n-2, n-1, 1)$ distribution.
\end{enumerate}
Conditions (a)--(c) are individually easy to satisfy,
but their \emph{simultaneous} satisfaction across $p - 1$
consecutive integers is a nontrivial structural constraint.

\subsection{Density of qualifying intervals}

The theorem applies whenever $p = 2n - 1$ is prime.
By the Prime Number Theorem, the number of primes of the
form $2n - 1$ up to $x$ is $\sim x / (2 \log x)$.
Thus the theorem governs a positive proportion of all
Legendre intervals, and the qualifying $n$ have no
essential gaps.

\subsection{Limitations}

We emphasize what the Spin Asymmetry Theorem does \emph{not}
prove:
it does not, by itself, establish that $\mathcal{I}_n$
contains a prime.
The quadratic residue structure modulo $p$ constrains the
\emph{algebraic shape} of the interval but does not directly
control the \emph{multiplicative structure} of individual
elements.
Converting the algebraic constraint into a proof of prime
existence would require additional input, potentially from
sieve theory or character sum estimates.

% ═══════════════════════════════════════════════════════════════════════
\section{Discrete Wronskian Volume and Radical Compression}
\label{sec:wronskian}
% ═══════════════════════════════════════════════════════════════════════

\subsection{Definition}

To quantify the multiplicative structure of prime-free
subsequences, we introduce the following measurement.

\begin{definition}[Discrete Wronskian compression]
\label{def:wronskian}
Let $C_1, C_2, \ldots, C_k$ be a maximal sequence of
consecutive composite numbers contained in $\mathcal{I}_n$.
The \emph{discrete Wronskian compression ratio} of this
sequence is
\begin{equation}\label{eq:wronskian}
    q_W
    \;=\;
    \frac{\displaystyle\sum_{i=1}^k \log C_i}
    {\displaystyle\log \rad\!\left(\prod_{i=1}^k C_i\right)}
    \;=\;
    \frac{\log\bigl(\prod C_i\bigr)}
    {\log \rad\bigl(\prod C_i\bigr)}.
\end{equation}
This ratio measures how much the prime factorizations of
the composites ``overlap'': shared prime factors reduce
the radical relative to the product.
\end{definition}

\begin{remark}
If all $C_i$ were squarefree and pairwise coprime,
then $\rad(\prod C_i) = \prod C_i$ and $q_W = 1$.
Values $q_W > 1$ indicate nontrivial sharing of prime factors
among the composites.
\end{remark}

\subsection{Experimental results}

For each target $n$, we located the longest maximal
prime-free sequence within $\mathcal{I}_n$ and computed
its Wronskian compression.

\begin{table}[H]
\centering
\caption{Discrete Wronskian compression of maximal
prime vacuums in Legendre intervals.}
\label{tab:wronskian}
\medskip
\begin{tabular}{@{}r r r r r@{}}
\toprule
$n$ & Max gap length &
$\log(\text{Volume})$ &
$\log(\text{Radical})$ &
$q_W$ \\
\midrule
$100$      & 27  & 248.64   & 179.26   & 1.3871 \\
$500$      & 39  & 484.68   & 371.44   & 1.3049 \\
$1{,}000$  & 55  & 759.78   & 607.50   & 1.2507 \\
$10{,}000$ & 103 & 1897.32  & 1546.25  & 1.2270 \\
$100{,}000$& 173 & 3983.47  & 3318.88  & 1.2002 \\
$200{,}000$& 247 & 6029.80  & 5006.12  & 1.2045 \\
\bottomrule
\end{tabular}
\end{table}

\subsection{Observations}

\textbf{(O1) Stabilization of $q_W$ near $1.20$.}
As $n$ increases across six orders of magnitude
(from $10^2$ to $2 \times 10^5$), the Wronskian
compression ratio decreases monotonically and stabilizes
in the range $1.20$--$1.21$ (Table~\ref{tab:wronskian}).
This is far below the individual Szpiro threshold $\sigma = 6$
and even below the squarefree baseline $\sigma = 3/2$.

\textbf{(O2) Sub-logarithmic gap growth.}
The maximal gap length grows as approximately
$k_{\max} \sim (\log n)^2$ (consistent with Cram\'er's
conjecture), while the compression ratio \emph{decreases}.
This means longer gaps do \emph{not} lead to more compressed
radicals; rather, longer sequences of composites incorporate
\emph{more} distinct prime factors, keeping the radical large.

\textbf{(O3) Self-regulation.}
The stabilization of $q_W$ suggests a self-regulating
mechanism: the natural distribution of primes prevents
the formation of long composite sequences whose
collective radical is highly compressed.
This is consistent with the heuristic that
``the primes are as regularly distributed as possible,
subject to divisibility constraints''
(cf.\ the Cram\'er model \cite{Cramer1936}).

\subsection{Comparison with the spin asymmetry}

The Spin Asymmetry Theorem (Section~\ref{sec:theorem})
provides an \emph{algebraic} structural constraint on
Legendre intervals, while the Wronskian compression
provides a \emph{multiplicative} constraint.
These two constraints operate on different axes:
\begin{itemize}[nosep]
\item \textbf{Algebraic axis:}
The Legendre symbol distribution is rigidly asymmetric,
with $N^- = N^+ + 1$.
\item \textbf{Multiplicative axis:}
The radical of consecutive composites is bounded from below,
with $q_W$ converging to $\approx 1.20$.
\end{itemize}
A proof of Legendre's Conjecture via this framework would
require showing that these two constraints are
\emph{jointly incompatible} with the absence of primes.

% ═══════════════════════════════════════════════════════════════════════
\section{Discussion}\label{sec:discussion}
% ═══════════════════════════════════════════════════════════════════════

\subsection{What is proven vs.\ what is observed}

To maintain clarity, we distinguish sharply between proven
results and empirical observations:

\emph{Proven} (Theorem~\ref{thm:main}):
Whenever $p = 2n - 1$ is prime, the interior of the
Legendre interval has exactly one more quadratic
non-residue than quadratic residue modulo $p$.
This is an unconditional algebraic identity.

\emph{Observed} (Table~\ref{tab:wronskian}):
The Wronskian compression of maximal prime vacuums
stabilizes near $q_W \approx 1.20$.
This is an empirical observation, not a theorem.

\emph{Conjectured}:
The combination of algebraic rigidity and multiplicative
self-regulation provides an obstruction to the existence of
prime-free Legendre intervals.
Formalizing this obstruction remains an open problem.

\subsection{Relation to character sum methods}

The Spin Asymmetry Theorem is closely related to classical
character sum estimates.
The excess of non-residues over residues in an interval
is measured by
\[
    \sum_{a \in \mathcal{I}_n^\circ} \Leg{a}{p}
    \;=\; N^+ - N^-
    \;=\; -1.
\]
This is a \emph{complete} character sum (over a full set of
$p - 1$ consecutive integers), which is why it evaluates
exactly.
Incomplete character sums over shorter sub-intervals
of $\mathcal{I}_n$ are governed by the P\'olya--Vinogradov
inequality and its refinements \cite{PV1918},
and connecting our result to these bounds is a natural
direction for future work.

\subsection{Open questions}

\begin{enumerate}[label=(\arabic*)]
\item \emph{Non-prime lengths.}
When $2n - 1$ is composite, the residue analysis modulo
$2n - 1$ becomes more complex (the Jacobi symbol replaces
the Legendre symbol, and the symmetry of QR/NR breaks down).
Can a useful structural theorem be obtained in this case?

\item \emph{Joint constraints.}
Can the algebraic constraint (spin asymmetry) and the
multiplicative constraint (Wronskian stabilization) be
combined into a single obstruction that rigorously
excludes prime vacuums?

\item \emph{Character sum refinement.}
The global identity $\sum \Leg{a}{p} = -1$ is exact but
``diffuse'' (it sums over the whole interval).
Do partial sums over sub-intervals of $\mathcal{I}_n$
exhibit additional structure beyond P\'olya--Vinogradov?

\item \emph{Higher-order residue symbols.}
Does the cubic or quartic residue symbol distribution
over $\mathcal{I}_n^\circ$ exhibit analogous asymmetries?
\end{enumerate}

\subsection*{Data and code availability}

All source code, experimental data, and the \LaTeX{} manuscript
are available at
\url{https://github.com/Ruqing1963/legendre-spin-asymmetry}.

% ═══════════════════════════════════════════════════════════════════════
\begin{thebibliography}{99}

\bibitem{ChenConductor}
  R.~Chen,
  ``Conductor incompressibility for Frey curves associated to
  prime gaps: rigidity obstructions to the Wiles paradigm in
  additive prime number theory,''
  Zenodo, 2026.
  \url{https://zenodo.org/records/18682375}

\bibitem{ChenLandau}
  R.~Chen,
  ``On Landau's fourth problem: conductor rigidity and
  Sato--Tate equidistribution for the $n^2+1$ family,''
  Zenodo, 2026.
  \url{https://zenodo.org/records/18683712}

\bibitem{ChenAP}
  R.~Chen,
  ``The 2-2 coincidence: conductor rigidity for primes in
  arithmetic progressions and the Bombieri--Vinogradov barrier,''
  Zenodo, 2026.
  \url{https://zenodo.org/records/18684151}

\bibitem{ChenFF}
  R.~Chen,
  ``The geometry of prime vacuums: Legendre's conjecture in
  function fields via monodromy and Chebotarev density,''
  Zenodo, 2026.

\bibitem{Cramer1936}
  H.~Cram\'er,
  ``On the order of magnitude of the difference between
  consecutive prime numbers,''
  \textit{Acta Arith.}\ \textbf{2} (1936), 23--46.

\bibitem{PV1918}
  G.~P\'olya,
  ``\"Uber die Verteilung der quadratischen Reste und Nichtreste,''
  \textit{G\"ott.\ Nachr.}\ (1918), 21--29;
  I.\,M.~Vinogradov,
  ``On a general theorem concerning the distribution of the
  residues and non-residues of powers,''
  \textit{Trans.\ Amer.\ Math.\ Soc.}\ \textbf{29} (1927), 209--217.

\end{thebibliography}

\end{document}
